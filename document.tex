\documentclass{cheatsheet}
\usepackage[utf8]{inputenc}
\usepackage[english]{babel}
\title{IT-Sicherheit und IT-Sicherheitsmanagement}
\IfFileExists{revision.tex}{\include{revision.tex}}{}
\begin{document}
    \section{Einführung}
    \begin{sectionbox}{Begrifflichkeiten \& Relevanz}
        \begin{itemize}
            \item ISMS: Teil eines Managementsystems
            \begin{itemize}
                \item Managementsystem: Rahmenwerk von Leitlinen, Verfahren, Richtlinen via Ressourcen
                \item Abwägung von Geschäftsrisiken nach Entwicklung, Implementierung, Durchführung, Überwachung, Überprüfung, Instandhaltung und Verbesserung der Informationssicherheit
            \end{itemize}
        \end{itemize}
        \begin{warningbox}{Schutzziele der Informationssicherheit}
            \begin{itemize}
                \item \textbf{C}onfidentiality: Geheimhaltung von Informationen
                \item \textbf{I}ntegrity:Integrität von Informationen
                \item \textbf{A}vailability: Verfügbarkeit von Informationen
            \end{itemize}
        \end{warningbox}
        \begin{itemize}   
            \item IT-Security-Pyramide: Top-Down-Ansatz via 
            \begin{itemize}
                \item ISP, Richtlinien (strategisch), Konzpete (operativ), Technologien (technisch)
                \item gelebte Prozess auf Basis von Doku
            \end{itemize}
            \item Asset: (im)materialle Werte, wie z.B. Informationen, Soft- \& Hardware, Mitarbeitende, Patente, Image
            \item Relevanz: Image, Kundenanforderung, IT-Angriffe, hohe Abhänigkeit von Geschäftsprozesse in IT
            \item Konsequenzen: rechliche \& regulative Anforderungen, interne oder Anforderungen gegenüber Providern
        \end{itemize}
        \begin{hintbox}{Beispiel: Erhöhung Branchenstandards (TISAX)}
            \begin{itemize}
                \item Grund: Cyber-Versicherungen erhöhen Anforderungen
                \item Öffentlichkeit regaiert sensibel --> hohe Relevanz 
            \end{itemize}
        \end{hintbox}
    \end{sectionbox}
    \begin{sectionbox}{Herausforderungen \& Lösungsansatz}
        \begin{itemize}
            \item Herausforderungen: ITSM angemessen?
            \begin{itemize}
                \item Erfüllung von rechtlichen Anforderungen? 
                \item Größte Risikofaktoren für Geschäftsrisiken?
            \end{itemize}
            \item Lösungsansatz: Orientierung an Standards
            \begin{itemize}
                \item breites Spektrum von Technik bis Management
                \item allgemein \& sektorspezifisch Zertifizierungen 
                \item Kostensenkung: Interoperalität, Vereinheilichung, Nachvollziehbarkeit
                \item Sicherheitsniveau: Aktualität durch zyklische (Wieder)-Bewertung, Stand der Technik
                \item Wettbewerbsvorteile: Zertifizierung, Rechtssicherheit, Image, Vergabeverfahren
            \end{itemize}
        \end{itemize}
    \end{sectionbox}
    \begin{sectionbox}{Einblick in rechtliche Grundlagen}
        \begin{itemize}
            \item KWG: Festlegung angemessen Notfallkonzeptes
            \item AktG: Überwachungssystem bezüglich Gefährdungen
            \item KRITIS: Energie, IT \& Telekommunikation, Ernährung, Wasser, Gesundheit, Finanzen \& Versicherungen, Transport \& Verkehr
        \end{itemize}
    \end{sectionbox}
\end{document}

